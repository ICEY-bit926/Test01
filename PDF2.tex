\documentclass[a4paper, 12pt]{article}
\setlength{\parindent}{0pt}
\usepackage[UTF8]{ctex}
\usepackage{xcolor}
\usepackage{fancyvrb}
\setlength{\parskip}{0pt}


\begin{document}
\title{实验报告}
\author{闫皓}
\date{\today}
\maketitle

\pagenumbering{roman}
\tableofcontents
\newpage
\pagenumbering{arabic}

\section{实验内容}
\subsection{Sheep}
(1) 实例一
\begin{Verbatim}
31595@ICEY MINGW64 /d/Desktop/ouc
$ foo=bar
31595@ICEY MINGW64 /d/Desktop/ouc
$ echo "$foo"
bar
31595@ICEY MINGW64 /d/Desktop/ouc
$ echo '$foo'
$foo
\end{Verbatim}

\texttt{{\color{blue}Bash 中的字符串通过 ' 和 " 分隔符来定义,但是它们的含义并不相同。以 ' 定义的字符串为原义字符串,其中的变量不会被转义,而 " 定义的字符串会将变量值进行替换。} }

(2)实例二
\begin{Verbatim}
31595@ICEY MINGW64 /d/Desktop
$ mcd () {
> mkdir -p "$1"
> cd "$1"
> }
31595@ICEY MINGW64 /d/Desktop
$ mcd hello
\end{Verbatim}

\texttt{{\color{blue}定义一个函数,它会创建一个文件夹并使用 cd 进入该文件夹。}}

(3)实例三
\begin{Verbatim}
31595@ICEY MINGW64 /d/Desktop/hello
$ false || echo "Oops, fail"
Oops, fail

31595@ICEY MINGW64 /d/Desktop/hello
$ true || echo "Will not be print"

31595@ICEY MINGW64 /d/Desktop/hello
$ true && echo "Things went well"
Things went well

31595@ICEY MINGW64 /d/Desktop/hello
$ false && echo "Will not be printed"

31595@ICEY MINGW64 /d/Desktop/hello
$ false ; echo "This will always run"
This will always run
\end{Verbatim}

{\color{blue}
\&\&(与操作符):只有前一个命令成功(即返回码为 0)时,才会执行后一个命令。\\
||(或操作符):只有前一个命令失败(即返回码非 0)时,才会执行后一个命令。\\
;(分号):无论前一个命令是否成功,分号后的命令都会执行。\\

1.false || echo "Oops, fail"\\
false 返回 1(非 0,表示失败),所以 || 右侧的 echo "Oops, fail" 被执行。\\
输出:Oops, fail\\

2.true || echo "Will not be printed"\\
true 返回 0(表示成功),所以 || 右侧的 echo 不执行。\\
输出:无输出\\

3.true \&\& echo "Things went well"\\
true 返回 0(表示成功),所以 \&\& 右侧的 echo "Things went well" 被执行。\\
输出:Things went well\\

4.false \&\& echo "Will not be printed"\\
false 返回 1(表示失败),所以 \&\& 右侧的 echo 不执行。\\
输出:无输出\\

5.false ; echo "This will always run"\\
false 返回 1(表示失败),但 ; 之后的命令不受影响,无论 false 成功与否,echo "This will always run" 都会执行。\\
输出:This will always run\\
}
(4)实例四
\begin{Verbatim}
31595@ICEY MINGW64 /d/Desktop/hello
$ echo "Starting program at $(date)"
Starting program at Fri Aug 30 16:14:01     2024

31595@ICEY MINGW64 /d/Desktop/hello
$ echo "Running program $0 with $# arguments with pid $$"
Running program /usr/bin/bash with 0 arguments with pid 683
\end{Verbatim}
{\color{blue}
命令替换 (Command Substitution):\\
\$(date) 将 date 命令的输出结果(当前日期和时间)替换到脚本中的对应位置。\\
示例:echo "Starting program at \$(date)" 会打印类似 "Starting program at Fri Aug 30 12:34:56 UTC 2024" 的输出。\\

变量解释:\\
\$0: 当前脚本的文件名。\\
\$\#: 传递给脚本的参数个数。\\
\$\$: 当前进程的 PID (进程 ID)。\\
}
(5)实例五
\begin{Verbatim}
31595@ICEY MINGW64 /d/Desktop/hello
$ for file in "$@"; do
> grep footbar "$file" > /dev/null 2> /dev/null
> if [[ $? -ne 0 ]]; then
> echo "File $file does not have any foobar, adding one"
> echo "# foobar" >> "$file"
> fi
> done
\end{Verbatim}
{\color{blue}
循环遍历参数:\\
for file in "\$@"; do ... done 会遍历所有传递给脚本的参数(每个参数通常是一个文件名)。\\

条件检查:\\
grep foobar "\$file" > /dev/null 2> /dev/null 会在当前文件中搜索字符串 "foobar"。输出和错误信息都被重定向到 /dev/null,所以它们不会显示在终端中。\\
if [[ \$? -ne 0 ]]; then ... fi 判断 grep 的退出状态。如果 grep 找不到字符串 "foobar",它会返回一个非零值,这意味着没有找到匹配。\\

追加注释:\\
echo "\# foobar" >> "\$file" 如果没有找到 "foobar",脚本会在文件末尾追加 \# foobar 这一行。\\
}
(6)实例六
\begin{Verbatim}
31595@ICEY MINGW64 /d/Desktop/hello
$ ls --help
\end{Verbatim}
{\color{blue}
查看“ls”的说明书。}

(7)实例七
\begin{Verbatim}
$ ls -laht --color=auto
total 7.0K
\end{Verbatim}
{\color{blue}
1.-l:使用长格式列出文件的详细信息(如权限、所有者、大小、修改时间等)。
2.-a:列出所有文件,包括隐藏文件(以 . 开头的文件)。
3.-h:以人类可读的格式显示文件大小(例如 454M 而不是 454279954)。
4.-t:按文件的最近访问时间排序(ls 默认按文件修改时间排序)。
5.--color=auto:以彩色显示输出(根据文件类型自动着色)。
}

(7)实例七
\begin{Verbatim}
# 保存当前工作目录
marco() {
    export MARCO_DIR="$(pwd)"
}

# 切换回保存的工作目录
polo() {
    if [ -n "$MARCO_DIR" ]; then
        cd "$MARCO_DIR" || echo "Failed to cd to saved directory"
    else
        echo "No directory saved. Run 'marco' first."
    fi
31595@ICEY MINGW64 /d/Desktop/hello
$ vim marco.sh
31595@ICEY MINGW64 /d/Desktop/hello
$ source marco.sh
31595@ICEY MINGW64 /d/Desktop/hello
$ marco
31595@ICEY MINGW64 /d/Desktop/hello
$ polo
}
\end{Verbatim}
{\color{blue}
marco 函数:\\
使用 pwd 获取当前工作目录,并将其存储在环境变量 MARCO\_DIR 中。\\
使用 export 命令确保 MARCO\_DIR 环境变量在当前 shell 会话中全局可用。\\

polo 函数:\\
首先检查 MARCO\_DIR 是否为空。如果不为空,执行 cd 命令切换到保存的目录。\\
如果 MARCO\_DIR 为空,说明还没有运行过 marco,因此提示用户需要先运行 marco。\\
}
(8)实例八
\begin{Verbatim}
#!/usr/bin/env bash

# 初始化变量
count=0
logfile="output.log"
error_logfile="error.log"

# 循环执行脚本,直到出错
while true; do
    ((count++))  # 增加运行次数
    ./your_script.sh >> "$logfile" 2>> "$error_logfile"  # 记录标准输出和错误输出到不同的文件
    
    # 检查上一个命令的退出状态
    if [[ $? -ne 0 ]]; then
        echo "Script failed after $count runs."
        echo "Standard Output:"
        cat "$logfile"  # 打印标准输出日志
        echo "Standard Error:"
        cat "$error_logfile"  # 打印错误输出日志
        break
    fi
done
\end{Verbatim}
{\color{blue}
变量初始化:\\
count=0:用于记录脚本执行的次数。\\
logfile="output.log":用于保存标准输出的日志文件。\\
error\_logfile="error.log":用于保存错误输出的日志文件。\\

循环执行:\\
使用 while true 使脚本在没有错误时无限循环执行。\\
每次循环中,count 变量增加,记录当前执行次数。\\
./your\_script.sh >> "\$logfile" 2>> "\$error\_logfile":执行脚本,将标准输出追加到 output.log,将标准错误流追加到 error.log。\\

检查退出状态:\\
if [[ \$? -ne 0 ]]; then:检查上一个命令的退出状态。如果返回值不是 0,说明脚本执行失败。\\
失败时,打印执行的次数,以及记录在日志文件中的标准输出和标准错误。\\
break 退出循环。\\

日志文件输出:\\
使用 cat 命令打印日志文件中的内容。\\
}
(9)实例九
\begin{Verbatim}
find . -type f -name "*.html" -print0 | xargs -0 zip html_files.zip
\end{Verbatim}
{\color{blue}
find . -type f -name "*.html" -print0:\\
find .:从当前目录(.)开始递归查找文件。\\
-type f:仅查找文件(不包括目录)。\\
-name "*.html":查找文件名以 .html 结尾的文件。\\
-print0:输出结果时,以空字符(null character)为分隔符,而不是默认的换行符。这对于处理文件名中包含空格或特殊字符的情况非常有用。\\

xargs -0 zip html\_files.zip:\\
xargs -0:读取 find 命令的输出,并使用空字符作为分隔符。-0 选项确保了文件名中的空格和特殊字符不会导致问题。\\
zip html\_files.zip:将所有找到的 HTML 文件压缩到一个名为 html\_files.zip 的 ZIP 文件中。\\
}
(10)实例十
\begin{Verbatim}
31595@ICEY MINGW64 /d/Desktop
$ mkdir ouc

31595@ICEY MINGW64 /d/Desktop
$ cd ouc

31595@ICEY MINGW64 /d/Desktop/ouc
$ touch hello.txt
\end{Verbatim}
{\color{blue}
在桌面建一个名为“ouc”的文件夹,将当前目录改为“ouc”在文件夹中创建一个名为“hello.txt”的文本文档。\\
}

(11)实例十一
\begin{Verbatim}
31595@ICEY MINGW64 /d/Desktop/ouc
$ echo hello > hello.txt

31595@ICEY MINGW64 /d/Desktop/ouc
$ cat hello.txt
hello

31595@ICEY MINGW64 /d/Desktop/ouc
$ cat < hello.txt > hello2.txt

31595@ICEY MINGW64 /d/Desktop/ouc
$ cat hello2.txt
hello

31595@ICEY MINGW64 /d/Desktop/ouc
$ rm hello2.txt
\end{Verbatim}
{\color{blue}
把“hello”输出到“hello.txt"文件当中,再输出”hello.txt“文件的内容。\\
把“hello.txt"文件的内容存入输出流中,在输出到”hello2.txt"文件当中,再将“hello.txt"文件的内容输出出来。\\
}

(12)实例十二
\begin{Verbatim}
# 删除全部扩展名为.tmp 的文件
find . -name '*.tmp' -exec rm {} ;
# 查找全部的 PNG 文件并将其转换为 JPG
find . -name '*.png' -exec convert {} {}.jpg ;
\end{Verbatim}

\subsection{Vim}
(1)实例一
\begin{Verbatim}
:%s/\<oldword\>/newword/gc
\end{Verbatim}
{\color{blue}
解释:将当前文件中所有出现的 oldword 替换为 newword,并在每次替换前提示用户确认。\\
}
(2)实例二
\begin{Verbatim}
:g/^$/d
\end{Verbatim}
{\color{blue}
解释:全局搜索空行并删除。\\
}
(3)实例三
\begin{Verbatim}
:g/string\_to\_delete/d
\end{Verbatim}
{\color{blue}
解释:全局搜索包含 string\_to\_delete 的所有行并删除。\\
}
(4)实例四
\begin{Verbatim}
:retab
\end{Verbatim}
{\color{blue}
解释:根据 expandtab 选项,将制表符转换为空格或相反操作。\\
}
(5)实例五
\begin{Verbatim}
:%s/\(.*\)/\L\1/
\end{Verbatim}
{\color{blue}
解释:将整个文件的每一行都转换为小写。\\
}
(6)实例六
\begin{Verbatim}
:%s/^\(.*\)\n\1$/\1/
\end{Verbatim}
{\color{blue}
解释:删除相邻的重复行(这是较简单的版本,复杂场景可能需要结合 sort 和 uniq 来处理)。\\
}
(7)实例七
\begin{Verbatim}
:w !diff % -
\end{Verbatim}
{\color{blue}
解释:使用 diff 比较当前文件与指定文件(可在新标签页中打开并手动替换差异部分)。\\
}
(8)实例八
\begin{Verbatim}
"ayiw
\end{Verbatim}
{\color{blue}
解释:将光标下的单词追加到 a 寄存器中,不会覆盖之前的内容。\\
}
(9)实例九
\begin{Verbatim}
:'<,'>!sort
\end{Verbatim}
{\color{blue}
解释:将选中的文本块传递给外部命令 sort,并用排序后的结果替换原内容。\\
}
(10)实例十
\begin{Verbatim}
:argdo %s/oldword/newword/gc | update
\end{Verbatim}
{\color{blue}
解释:在所有被缓冲的文件中查找并替换 oldword 为 newword,并保存修改。\\
}
(11)实例十一
\begin{Verbatim}
:%s/\<word_to_count\>//gn
\end{Verbatim}
{\color{blue}
解释:统计并显示 word\_to\_count 在当前文件中出现的次数。\\
}
(12)实例十二
\begin{Verbatim}
:diffget /path/to/other/file
\end{Verbatim}
{\color{blue}
解释:在 diff 模式下,从另一个文件中获取并合并差异。\\
}

\section{实验感悟}
在今天的实验课上,我深入学习了Sheep和Vim两个工具。通过这次学习,我对它们的功能和应用有了更深的理解。\\

首先是Sheep,它是一种极具效率的工具,能够帮助我们更好地管理和处理资源。通过实验操作,我体会到了Sheep在并行计算和资源调度中的重要性。它的灵活性和强大的功能,使得复杂的计算任务得以顺利完成。在实际操作中,我尝试了不同的配置,了解了Sheep在不同环境下的表现。这让我意识到,熟练掌握Sheep的配置和优化技巧,能大大提高工作效率。\\

接着是Vim,这是一款经典的文本编辑器,虽然在使用初期感到有些复杂,但通过今天的练习,我逐渐掌握了它的基本命令和操作技巧。Vim的高效编辑模式和丰富的插件支持,给我留下了深刻的印象。尤其是在处理大规模文本和代码时,Vim的快捷操作极大地提升了我的编辑速度。在学习过程中,我还掌握了如何通过Vim进行查找和替换、如何使用寄存器、以及如何进行多文件编辑等高级技巧。\\

\end{document}
