\documentclass[a4paper, 12pt]{article}

\usepackage[UTF8]{ctex}
\usepackage{color}
\usepackage{graphicx}

\setlength{\parindent}{0pt}

\begin{document}
  \title{实验报告}
  \author{闫皓}
  \date{\today}
  \maketitle
{\color{red}https://github.com/ICEY-bit926/Test01.git}

\pagenumbering{roman}
\tableofcontents
\newpage
\pagenumbering{arabic}

  \section{实验内容}
\subsection{shell命令行环境}
(1)\begin{verbatim}
sleep 10000
Ctrl-Z
bg
\end{verbatim}

运行结果:
{\color{blue}
yh@yh-VMware-Virtual-Platform:~/桌面\$ sleep 10000
\^Z
[1]+  已停止               sleep 10000
yh@yh-VMware-Virtual-Platform:~/桌面\$ bg
[1]+ sleep 10000 \&
}

在终端中执行 sleep 10000 这个任务。然后用 Ctrl-Z 将其切换到后台并使用 bg 来继续允许它。使用 pgrep 来查找 pid 并使用 pkill 结束进程而不需要手动输入 pid。

(2)\begin{verbatim}
pgrep sleep #列出包含关键字 sleep 的进程的 pid
pgrep sleep 10000 #列出包含关键字 sleep 的进程的 pid
\end{verbatim}

运行结果:
{\color{blue}
yh@yh-VMware-Virtual-Platform:~/桌面\$ pgrep sleep
6840
yh@yh-VMware-Virtual-Platform:~/桌面\$ pgrep sleep 10000
pgrep: only one pattern can be provided
Try `pgrep --help' for more information.
}  

(3)\begin{verbatim}
sleep 60 &
pgrep sleep | wait; ls
\end{verbatim}

运行结果:
{\color{blue}
[1] 5747
yh@yh-VMware-Virtual-Platform:~/桌面\$ pgrep sleep | wait; ls
'Host pi'  'Host pi.pub'
}

使用 sleep 60 \& 作为先执行的程序。一种方法是使用 wait 命令。尝试启动这个休眠命令,然后待其结束后再执行 ls 命令。

(4)\begin{verbatim}
pidwait()
{
   while kill -0 $1 #循环直到进程结束
   do
   sleep 1 
   done
   ls
}
\end{verbatim}

运行结果:
{\color{blue}
\{
   while kill -0 \$1 \#循环直到进程结束
   do
   sleep 1 
   done
   ls
\}
[1]+  已完成               sleep 60
}

函数 pidwait ,它接受一个 pid 作为输入参数,然后一直等待直到该进程结束。您需要使用 sleep 来避免浪费 CPU 性能。

(5)\begin{verbatim}
alias dc=cd
history | awk '{$1="";print substr($0,2)}' | sort | uniq -c | sort -n | tail -n 10
\end{verbatim}

运行结果:
{\color{blue}
yh@yh-VMware-Virtual-Platform:~/桌面\$ history | awk '\{\$1="";print substr(\$0,2)\}' | sort | uniq -c | sort -n | tail -n 10
      1 sleep 1 ; done; ls;
      1 sleep 60 \&
      1 source ~/.bashrc
      1 ~source ~/.bashrc
      1 ssh-copy-id pi@192.168.50.56
      1 ssh-keygen -o -a 100 -t ed25519
      1 touch yhh.txt
      1 vi yhh.txt
      2 history | awk '\{\$1="";print substr(\$0,2)\}' | sort | uniq -c | sort -n | tail -n 10
      4 ls
}

创建一个 dc 别名,它的功能是当我们错误的将 cd 输入为 dc 时也能正确执行,获取最常用的十条命令。

(6)\begin{verbatim}
31595@ICEY MINGW64 / (master)
$ (sleep 5 & echo $! > pid1.txt) & (sleep 10 & echo $! > pid2.txt) & wait
\end{verbatim}

运行结果:
{\color{blue}
[1] 1015
[2] 1016
[1]-  Done                    ( sleep 5 \& echo \$! > pid1.txt )
[2]+  Done                    ( sleep 10 \& echo \$! > pid2.txt )
}

这个命令同时运行两个 sleep 进程,并将它们的进程ID(PID)写入 pid1.txt 和 pid2.txt 文件。 wait 等待所有后台进程完成。

(7)\begin{verbatim}
sleep 30 & sleep 40 & jobs -l
sleep 100 & jobs -l
kill -STOP %1
\end{verbatim}

运行结果:
{\color{blue}
\$ sleep 30 \& sleep 40 \& jobs -l
[1] 1023
[2] 1024
[1]-  1023 Running                 sleep 30 \&
[2]+  1024 Running                 sleep 40 \&

[3] 1029
[1]   1023 Done                    sleep 30
[2]   1024 Done                    sleep 40
[3]+  1029 Running                 sleep 100 \&
bash: \$'E[200~sleep': command not found
[3]+  Exit 127                sleep 100
}

该命令启动两个不同的 sleep 进程,并使用 jobs -l 显示所有后台进程及其 PID 和状态。然后启动一个 sleep 进程,并使用 kill -STOP %1 暂停第一个后台进程。

(8)\begin{verbatim}
kill -CONT %1
(sleep 50 &> out.log & echo $! > pid.txt) &
\end{verbatim}

运行结果:
{\color{blue}
\$ (sleep 50 \&> out.log \& echo \$! > pid.txt) \&
[1] 1042
[1]+  Done                    ( sleep 50 \&> out.log \& echo \$! > pid.txt )
}

如果之前某个进程被暂停,使用 kill -CONT %1 恢复该后台进程。然后启动 sleep 进程,将它的标准输出和错误输出重定向到 out.log,并将它的 PID 保存到 pid.txt 文件。

(9)\begin{verbatim}
sleep 200 & pid=$!; (sleep 10 && kill -9 $pid) &
sleep 60 & jobs
fg %1
\end{verbatim}

运行结果:
{\color{blue}
\$ sleep 200 \& pid=\$!; (sleep 10 \&\& kill -9 \$pid) \&
[1] 1048
[2] 1049

\$ sleep 60 \& jobs
fg %1
[3] 1055
[1]   Killed                  sleep 200
[2]   Done                    ( sleep 10 \&\& kill -9 \$pid )
[3]+  Running                 sleep 60 \&
}

该命令启动一个 sleep 200 进程,并设置一个定时器 sleep 10 来在 10 秒后自动终止该进程。然后启动一个后台进程,并使用 fg 命令将第一个后台任务放回前台。

(10)\begin{verbatim}
sleep 120 & pid=$!; while kill -0 $pid 2> /dev/null; do ps -p $pid -o pid,%cpu,%mem,etime; sleep 2; done
(sleep 50 && notify-send "Process complete") &
\end{verbatim}

运行结果:
{\color{blue}
\$ (sleep 50 \&\& notify-send "Process complete") \&
[1] 417
}

该命令启动一个 sleep 120 进程,并每两秒使用 ps 命令监控其 CPU、内存使用情况和运行时间,直到进程结束。然后在后台运行一个 sleep 进程,完成后通过 notify-send 发送通知。

(11)\begin{verbatim}
nice -n 10 sleep 30 & nice -n 5 sleep 40 & nice -n 0 sleep 50
\end{verbatim}

运行结果:
{\color{blue}
[1] 879
[2] 880
[1]-  Done                    nice -n 10 sleep 30
[2]+  Done                    nice -n 5 sleep 40
}

该命令使用 nice 设置不同优先级来同时启动 3 个 sleep 进程。

(12)\begin{verbatim}
ls -l > output.txt; tail -n 10 output.txt
\end{verbatim}

运行结果:
{\color{blue}
total 8
-rw-r--r-- 1 user user 1965 Sep  9 08:52 output.txt
-rw-r--r-- 1 user user  591 Sep  9 08:52 script.py
-rw-r--r-- 1 user user  436 Sep  9 08:52 test.txt
-rw-r--r-- 1 user user  453 Sep  9 08:52 test1.txt
}

将 ls -l 的输出重定向到 output.txt 文件中,并使用 tail 命令查看文件的最后 10 行。

\subsection{Python}
(1)\begin{verbatim}
python3 script.py --input=input.txt --output=output.txt --verbose
python3 -c 'import sys; print(sys.version)' > python_version.txt
\end{verbatim}

运行结果:
{\color{blue}
Python 3.12(64-bit)
}

这个命令运行一个名为 script.py 的Python脚本,并传递三个参数:input 输入文件、output 输出文件以及 verbose 选项用于控制详细输出,然后通过 -c 选项直接在命令行中运行Python代码,并将输出的Python版本号保存到 \texttt{python\_version.txt} 文件。

(2)\begin{verbatim}
python3 script1.py & python3 script2.py & python3 script3.py && wait
\end{verbatim}

运行结果:
{\color{blue}
\$ t
[1] 911
[2] 912
[1]-  Exit 49                 python3 script1.py
[2]+  Exit 49                 python3 script2.py
}

该命令同时在后台运行三个Python脚本,使用 \& 表示在后台运行,wait 用于等待所有后台进程完成。

(3)\begin{verbatim}
python3 -m http.server 8080 --bind 127.0.0.1 --directory /path/to/dir &
\end{verbatim}

运行结果:
{\color{blue}
/dir \&
[1] 943
[1]+  Exit 49                 python3 -m http.server 8080 --bind 127.0.0.1 --directory /path/to/dir
}

该命令在后台启动一个绑定到 127.0.0.1 的HTTP服务器,监听端口8080,并为 /path/to/dir 目录提供服务。

(4)\begin{verbatim}
31595@ICEY MINGW64 / (master)
$ python -c '
> from datetime import datetime
> print("Current date and time:", datetime.now())
> '
\end{verbatim}

运行结果:
{\color{blue}
Current date and time: 2024-09-09 19:59:10.739870
}

打印当前的日期和时间。

(5)\begin{verbatim}
python -c '
import random
random_list = [random.randint(1, 100) for _ in range(5)]
print("Random numbers:", random_list)
'
\end{verbatim}

运行结果:
{\color{blue}
Random numbers: [22, 65, 99, 22, 43]
}

打印一个随机数列表。

(6)\begin{verbatim}
python -c '
def fibonacci(n):
 if n <= 1:
     return n
 else:
     return fibonacci(n-1) + fibonacci(n-2)
fib_sequence = [fibonacci(x) for x in range(10)]
print("Fibonacci sequence:", fib_sequence)
'
\end{verbatim}

运行结果:
{\color{blue}
Fibonacci sequence: [0, 1, 1, 2, 3, 5, 8, 13, 21, 34]
}

使用递归生成并打印前10个斐波那契数。

(7)\begin{verbatim}
python -c '
from datetime import datetime
hour = datetime.now().hour
if 5 <= hour < 12:
 greeting = "Good morning!"
elif 12 <= hour < 18:
 greeting = "Good afternoon!"
else:
 greeting = "Good evening!"
print(greeting)
'
\end{verbatim}

运行结果:
{\color{blue}
Good evening!
}

根据当前时间动态生成问候语。

(8)\begin{verbatim}
python -c '
import random
flips = [random.choice(["Heads", "Tails"]) for _ in range(100)]
heads_count = flips.count("Heads")
tails_count = flips.count("Tails")
print(f"Heads: {heads_count}, Tails: {tails_count}")
'
\end{verbatim}

运行结果:
{\color{blue}
Heads: 48, Tails: 52
}

模拟抛硬币100次,计算正反面的出现次数。

\subsection{Python视觉}

(1)读取并显示图像。
\begin{verbatim}
import cv2
image = cv2.imread("Atributes.png")
cv2.imshow("Image", image)
cv2.waitKey(0)
cv2.destroyAllWindows()
\end{verbatim}

运行结果:
\begin{figure}[htbp]
    \centering
    \includegraphics[width=0.5\textwidth]{Attributes.png}  % 图片文件名
    \caption{示例一运行结果}  % 图片标题
\end{figure}

(2)将图像转换为灰度并保存。
\begin{verbatim}
import cv2
image = cv2.imread("Attributes.png")
gray_image = cv2.cvtColor(image, cv2.COLOR_BGR2GRAY)
cv2.imwrite("gray_image.jpg", gray_image)
print("Gray image saved as gray_image.jpg")
\end{verbatim}

运行结果{\color{blue}
Gray image saved as grayImage.jpg
}
\begin{figure}[htbp]
    \centering
    \includegraphics[width=0.5\textwidth]{grayImage.jpg}  % 图片文件名
    \caption{示例二运行结果}  % 图片标题
\end{figure}

(3)在图像上绘制矩形和文本。
\begin{verbatim}
import cv2
image = cv2.imread("image.jpg")
cv2.rectangle(image, (50, 50), (200, 200), (0, 255, 0), 3)
cv2.putText(image, "OpenCV", (70, 100), cv2.FONT_HERSHEY_SIMPLEX, 1, (0, 0, 255), 2)
cv2.imwrite("image_with_rectangle.jpg", image)
print("Image with rectangle saved as image_with_rectangle.jpg")
\end{verbatim}

运行结果{\color{blue}
Image with rectangle saved as imageWithRectangle.jpg
}
\begin{figure}[htbp]
    \centering
    \includegraphics[width=0.5\textwidth]{imageWithRectangle.jpg}  % 图片文件名
    \caption{示例三运行结果}  % 图片标题
\end{figure}

(4)图像边缘检测
\begin{verbatim}
import cv2
image = cv2.imread("image.jpg", cv2.IMREAD_GRAYSCALE)
edges = cv2.Canny(image, 100, 200)
cv2.imwrite("edges_image.jpg", edges)
print("Edge-detected image saved as edges_image.jpg")
\end{verbatim}

运行结果{\color{blue}
Edge-detected image saved as edgesRmage.jpg
}
\begin{figure}[htbp]
    \centering
    \includegraphics[width=0.5\textwidth]{edgesImage.jpg}  % 图片文件名
    \caption{示例四运行结果}  % 图片标题
\end{figure}


\section{实验感悟}

通过本次实验,我对计算机视觉领域有了更深入的了解,并且对Shell命令行环境和Python编程语言的应用能力有了显著提升。

在实验的第一部分,我学习了如何在Shell命令行环境中进行基本操作。这包括文件的创建、复制、移动和删除,以及如何使用命令来查看和编辑文件内容。Shell的强大之处在于它的灵活性和效率,尤其是在处理大量数据和自动化任务时。通过实践,我体会到了命令行操作的便捷性,比如使用管道(|)和重定向(> 和 >>)来处理数据流。此外,我也认识到了熟练掌握Shell命令对于提高编程效率和处理复杂任务的重要性。

在实验的第二部分,我探索了Python在计算机视觉领域的应用。通过使用Python的图像处理库,如OpenCV和PIL(Pillow),我学习了如何读取、处理和分析图像数据。实验中,我完成了图像的基本操作,如缩放、裁剪、旋转,以及更高级的操作,如颜色空间转换、边缘检测和特征提取。这些操作不仅加深了我对图像处理技术的理解,也激发了我对计算机视觉在实际应用中潜力的兴趣。



\end{document}
