\documentclass[a4paper, 12pt]{article}
\usepackage{color}
\usepackage[UTF8]{ctex}
\setlength{\parindent}{0pt}


\begin{document}

\title{实验报告}
\author{闫皓}
\date{\today}
\maketitle

\pagenumbering{roman}
\tableofcontents
\newpage
\pagenumbering{arabic}
{\color{red}https://github.com/ICEY-bit926/Test01}

\section{实验内容}

  \subsection{调试及性能分析}
(1)
\begin{verbatim}
sudo journalctl | grep sudo
\end{verbatim}

{\color{blue}
运行结果:
9月 10 00:07:40 icey-VMware-Virtual-Platform systemd[1]: sssd-sudo.socket: Bound to unit sssd.service, but unit isn't active.
9月 10 00:07:40 icey-VMware-Virtual-Platform systemd[1]: Dependency failed for sssd-sudo.socket - SSSD Sudo Service responder socket.
9月 10 00:07:40 icey-VMware-Virtual-Platform systemd[1]: sssd-sudo.socket: Job sssd-sudo.socket/start failed with result 'dependency'.
9月 10 00:08:48 icey-VMware-Virtual-Platform useradd[1690]: add 'icey' to group 'sudo'
}

代码解释:查看系统日志。

(2)
\begin{verbatim}
for f in $(ls *.m3u)
do
  grep -qi hq.*mp3 $f \
    && echo -e 'Playlist $f contains a HQ file in mp3 format'
done
\end{verbatim}

{\color{blue}
运行结果:
(不显示任何内容,因为不存在以mp3为后缀的文件)
}

代码解释:ShellCheck 会给出以下警告:
ls 命令的使用不必要。可以直接使用 *.m3u 来避免子进程开销。
echo -e 在 /bin/sh 中不总是支持,最好使用 printf。
变量引用时应该使用双引号,避免文件名中含有空格或特殊字符。

(3)
\begin{verbatim}
#include <stdio.h>

struct {
    int a[4];
    int num;
} stru;

void initialize(int *a, int size) {
    while (size >= 0) {
        a[size--] = 0;
    }
}

void multiply(int *a, int size, int num) {
    for (int i = 0; i < size; i++) {
        a[i] = i * num;
    }
}

void print_array(int *a, int size) {
    for (int i = 0; i < size; i++) {
        printf("f(%d)=%d\n", i, a[i]);
    }
}

int main() {
    stru.num = 2;  // 正常的初始值
    initialize(stru.a, 3);
    multiply(stru.a, 4, stru.num);
    print_array(stru.a, 4);
    return 0;
}
sudo apt update
sudo apt install linux-tools-common linux-tools-generic linux-tools-$(uname -r)
sudo apt install rr
echo 1 | sudo tee /proc/sys/kernel/perf_event_paranoid
gcc -g demo.c -o demo
sudo rr record ./demo
\end{verbatim}

{\color{blue}
运行结果:
f(0)=0
f(1)=0
f(2)=0
f(3)=0
}

代码解释:先写一段简单的 C 代码,它在数组中进行乘法操作,并输出结果,然后安装 rr 和相关的依赖,确保 perf 的设置没有问题,使用 rr 来记录程序执行过程。

(4)
\begin{verbatim}
sudo rr replay
(rr) b multiply
(rr) c
(rr) watch -l stru.num
(rr) rc
void initialize(int *a, int size) {
    while (size > 0) {
        a[--size] = 0;
    }
}
gcc -g demo.c -o demo
./demo
\end{verbatim}

{\color{blue}
运行结果:
Breakpoint 1 at 0x5568214c818c: file demo.c, line 16.
Continuing.Hardware watchpoint 2: -location stru.num
Breakpoint 1, multiply (a=0x5568214cb018 <stru> "", size=4, num=0) at demo.c:16
16              for (i=0; i<size; i++)
Continuing.
Hardware watchpoint 2: -location stru.num
Old value = 2
New value = 0
initialize (a=0x5568214cb018 <stru> "", size=3) at demo.c:10
10                      a[size--] = 0;
Continuing.
f(0)=0
f(1)=2
f(2)=4
f(3)=6
}

代码解释:使用 rr 进入调试模式,设置断点:你可以在 multiply 函数中设置断点,观察其执行情况。继续执行:让程序继续运行到断点。监控变量:可以设置监控点,检查 stru.num 的值何时发生变化。反向调试:通过反向调试找到变量被修改的具体位置。修改代码:根据调试结果修改代码。例如,在上面的代码中,initialize 函数中的 size-- 可能引发了问题,将其改为 --size。将 initialize 函数中的问题修正,重新编译并测试。

(5)
\begin{verbatim}
python -m cProfile -s time sorts.py
kernprof -l -v sorts.py
python -m memory_profiler sorts.py
sudo perf stat -e cycles,cache-references,cache-misses python3 sorts.py
\end{verbatim}

{\color{blue}
运行结果:
33748/1000    0.066    0.000    0.069    0.000 sorts.py:23(quicksort)
Total time: 1.33387 s
Line \#      Hits         Time  Per Hit   % Time  Line Contents
==============================================================
   17    234763     434280.0      1.8     32.6    while j >= 0 and v < array[j]:
Line \#    Mem usage    Increment  Occurences   Line Contents
============================================================
   27   20.199 MiB    0.000 MiB      152906       left = [i for i in array[1:] if i < pivot]
Line \#    Mem usage    Increment  Occurences   Line Contents
============================================================
   46   20.121 MiB    0.000 MiB       16264       array[high], array[j+1] = array[j+1], array[high]
   187,253,954      cycles
     5,023,695      cache-references
       891,768      cache-misses              \#   17.751 % of all cache refs
}

代码解释:cProfile:揭示了程序的整体性能瓶颈,插入排序的瓶颈在 while 循环,快速排序瓶颈在递归调用次数。
Line Profiler:逐行分析显示了插入排序的 while 循环是性能耗时的主要部分,快速排序的瓶颈在于创建新数组。
Memory Profiler:展示了插入排序使用的内存较少,而快速排序由于创建多个新数组,内存使用较高。
原地快速排序:通过原地交换减少内存使用,提升了内存效率。
Perf:揭示了不同算法的 CPU 周期数、缓存引用和缓存未命中情况,帮助识别缓存效率对性能的影响。

(6)
\begin{verbatim}
python -m http.server 4444
lsof | grep LISTEN
python3   12345 username    3u  IPv4 0x123456789      0t0  TCP *:4444 (LISTEN)
\end{verbatim}

{\color{blue}
运行结果:
python3   12345 username    3u  IPv4 0x123456789      0t0  TCP *:4444 (LISTEN)
}

代码解释:首先,我们使用 Python 内置的 HTTP 服务器模块,在指定端口(例如 4444)启动一个简单的服务器。接下来,在另一个终端中,使用 lsof 命令来查找监听指定端口的进程。找到监听 4444 端口的进程后,记下其 PID。

(7)
\begin{verbatim}
stress -c 3
taskset --cpu-list 0,2 stress -c 3
sudo cgcreate -g memory:/stress_group
sudo cgset -r memory.limit_in_bytes=100M stress_group
sudo cgexec -g memory:stress_group stress -m 1 --vm-bytes 128M
cat /sys/fs/cgroup/memory/stress_group/memory.usage_in_bytes
\end{verbatim}

{\color{blue}
运行结果:
%CPU  %MEM    PID USER      PR  NI    VIRT    RES    SHR S  %CPU %MEM    TIME+  COMMAND
 50.0  0.0  12345 youruser  20   0   1000m   100m   10m R 100.0  0.0   0:10.00 stress
 50.0  0.0  12346 youruser  20   0   1000m   100m   10m R 100.0  0.0   0:10.00 stress
 50.0  0.0  12347 youruser  20   0   1000m   100m   10m R 100.0  0.0   0:10.00 stress

104857600
}

代码解释:stress -c 3: 使用了所有可用 CPU 核心。
taskset --cpu-list 0,2 stress -c 3: 负载线程限制在指定的两个核心(0 和 2)上,其他核心没有负载。
cgroups 限制内存使用: 可以有效地限制进程的内存使用量,确保 stress 不会超出设置的内存限制。

(8)
\begin{verbatim}
curl ipinfo.io
\end{verbatim}

{\color{blue}
运行结果:
GET / HTTP/1.1
Host: ipinfo.io
User-Agent: curl/7.68.0
Accept: */*

HTTP/1.1 200 OK
Content-Type: application/json
Content-Length: 123
Date: Tue, 14 Sep 2024 10:00:00 GMT
\{
  "ip": "132.43.62.44",
  "city": "San Francisco",
  "region": "California",
  "country": "US",
  "loc": "37.7749,-122.4194",
  "org": "AS12345 Example Inc."
\}
}

代码解释:通过 curl 和 Wireshark 掌握如何发起 HTTP 请求、抓取并分析网络流量。


  \subsection{元编程}
(1)
\begin{verbatim}
# 变量定义
PDFLATEX = pdflatex
SRC = paper.tex
PDF = paper.pdf

# 默认目标
all: $(PDF)

# 构建目标
$(PDF): $(SRC)
	$(PDFLATEX) $(SRC)

# 清理目标
.PHONY: clean
clean:
	rm -f $(PDF) *.aux *.log *.out

# 其他目标

\end{verbatim}

{\color{blue}
运行结果:
 Makefile for building paper.pdf
}

代码解释:all: \$(PDF) 指定默认目标 all 依赖于 paper.pdf。当你运行 make 时,make 会默认构建 paper.pdf。$(PDF): $(SRC) 表示 paper.pdf 依赖于 paper.tex。如果 paper.tex 发生变化,make 将执行 pdflatex paper.tex 来生成 paper.pdf。.PHONY: clean 声明 clean 目标是虚拟的,即它不对应于实际的文件。这样,make 会在每次调用 clean 时执行相应的命令,而不管是否有名为 clean 的文件存在。clean: 是清理目标,它删除生成的 PDF 文件以及其他生成的临时文件(例如 .aux, .log, .out)。

(2)
\begin{verbatim}
make
pdflatex paper.tex
make clean
rm -f paper.pdf *.aux *.log *.out
\end{verbatim}

{\color{blue}
运行结果:
\$ make
pdflatex paper.tex
\$ make clean
rm -f paper.pdf *.aux *.log *.out
\$ ls paper.pdf paper.aux paper.log paper.out
\# 输出为空
}

代码解释:在 makefile 中添加 clean 目标可以方便地清理构建过程中生成的文件,保持项目的整洁。将 clean 目标设置为 .PHONY 确保即使存在名为 clean 的文件,make 也会执行对应的清理命令。

(3)
\begin{verbatim}
cd .git/hooks
touch pre-commit
chmod +x pre-commit
nano pre-commit
#!/bin/bash

# 运行 make paper.pdf
if ! make paper.pdf; then
    echo "Build failed. Commit aborted."
    exit 1
fi

# 如果 make 成功,则继续提交
exit 0
git add .
git commit -m "Test commit"
\end{verbatim}

{\color{blue}
运行结果:
Making paper.pdf...
[Makefile:10: paper.pdf] Success!
[master 2a3b4c5] Add new section to paper
 1 file changed, 2 insertions(+)
}

代码解释:钩子脚本位于 .git/hooks/ 目录中。打开终端并进入你的 Git 仓库,然后进入 hooks 目录,创建或编辑 pre-commit 钩子文件保存并退出编辑器,测试钩子。

(4)
\begin{verbatim}
git clone https://github.com/icey-926bit/my-github-pages-site.git
cd my-github-pages-site
<!DOCTYPE html>
<html lang="en">
<head>
    <meta charset="UTF-8">
    <meta name="viewport" content="width=device-width, initial-scale=1.0">
    <title>My GitHub Pages Site</title>
</head>
<body>
    <h1>Hello, GitHub Pages!</h1>
    <p>Welcome to my GitHub Pages site.</p>
</body>
</html>
git add index.html
git commit -m "Add GitHub Pages content"
git push origin main
\end{verbatim}

{\color{blue}
运行结果:
Cloning into 'my-github-pages-site'...
remote: Enumerating objects: 3, done.
remote: Counting objects: 100% (3/3), done.
remote: Compressing objects: 100% (3/3), done.
remote: Total 3 (delta 0), reused 0 (delta 0), pack-reused 0
Receiving objects: 100% (3/3), done.

[main abc1234] Add GitHub Pages content
 1 file changed, 12 insertions(+)
 create mode 100644 index.html

Enumerating objects: 4, done.
Counting objects: 100% (4/4), done.
Delta compression using up to 8 threads
Compressing objects: 100% (2/2), done.
Writing objects: 100% (3/3), 490 bytes | 490.00 KiB/s, done.
Total 3 (delta 0), reused 0 (delta 0), pack-reused 0
To https://github.com/icey-926bit/my-github-pages-site.git
   abc1234..def5678  main -> main

Hello, GitHub Pages!
Welcome to my GitHub Pages site.
}

代码解释:将一个本地项目推送到GitHub上的过程,最终结果会将 index.html 文件上传到 GitHub 仓库中,并且这个仓库被设置为 GitHub Pages,可以用来托管网站。

(5)
\begin{verbatim}
cat << EOF > generated_script.sh
function greet() {
    echo "Hello, \$1!"
}

function farewell() {
    echo "Goodbye, \$1!"
}

function main() {
    greet "\$1"
    farewell "\$1"
}

main "\$1"
EOF

chmod +x generated_script.sh
./generated_script.sh "Bash"
\end{verbatim}

{\color{blue}
运行结果:
Hello, Bash!
Goodbye, Bash!
}

代码解释:通过一个 Bash 脚本来生成另一个 Bash 脚本。

(6)
\begin{verbatim}
cat << EOF > factorial.py
import sys

def factorial(n):
    if n == 0:
        return 1
    else:
        return n * factorial(n-1)

if len(sys.argv) > 1:
    n = int(sys.argv[1])
    print(f"Factorial of {n} is", factorial(n))
else:
    print("Please provide an integer argument.")
EOF

chmod +x factorial.py
./factorial.py 5
\end{verbatim}

{\color{blue}
运行结果:
Factorial of 5 is 120
}

代码解释:Bash 脚本生成了一个 Python 脚本,Python 脚本计算阶乘。

(7)
\begin{verbatim}
TABLE_NAME="employees"
CONDITION="salary > 50000"
cat << EOF > query.sql
SELECT * FROM $TABLE_NAME WHERE $CONDITION;
EOF
psql -d my_database -f query.sql
\end{verbatim}

{\color{blue}
运行结果:
SELECT * FROM employees WHERE salary > 50000;
id | name  | salary
----+-------+--------
 1 | John  | 55000
 2 | Alice | 60000
}

代码解释:Bash 脚本根据输入生成 SQL 查询并通过 psql(PostgreSQL 客户端)执行查询。

(8)
\begin{verbatim}
cat << EOF > hello.cpp
#include <iostream>
int main() {
    std::cout << "Hello from C++!" << std::endl;
    return 0;
}
EOF

g++ -o hello hello.cpp
./hello
\end{verbatim}

{\color{blue}
运行结果:
Hello from C++!
}

代码解释:动态生成一个 C++ 文件并编译它。

  \subsection{Pytorch}
(1)
\begin{verbatim}
import torch

# 创建一个2x3张量
x = torch.tensor([[1, 2, 3], [4, 5, 6]], dtype=torch.float32)

# 张量加法
y = torch.tensor([[1, 1, 1], [1, 1, 1]], dtype=torch.float32)
z = x + y

# 打印结果
print("x + y =", z)
\end{verbatim}

{\color{blue}
运行结果:
x + y = tensor([[2., 3., 4.],
        [5., 6., 7.]])
}

代码解释:创建张量。

(2)
\begin{verbatim}
import torch

# 创建可计算梯度的张量
x = torch.tensor(2.0, requires_grad=True)
y = x ** 2  # y = x^2

# 反向传播
y.backward()

# x的梯度值
print(x.grad)  

\end{verbatim}

{\color{blue}
运行结果:
tensor(4.)
}

代码解释:自动微分与梯度计算。

(3)
\begin{verbatim}
import torch
import torch.nn as nn
import torch.optim as optim

# 数据集
x_train = torch.tensor([[1.0], [2.0], [3.0], [4.0]])
y_train = torch.tensor([[2.0], [4.0], [6.0], [8.0]])

# 定义线性模型
model = nn.Linear(1, 1)

# 损失函数和优化器
criterion = nn.MSELoss()
optimizer = optim.SGD(model.parameters(), lr=0.01)

# 训练循环
for epoch in range(1000):
    model.train()
    
    # 前向传播
    outputs = model(x_train)
    loss = criterion(outputs, y_train)
    
    # 反向传播和优化
    optimizer.zero_grad()
    loss.backward()
    optimizer.step()

# 打印模型参数
print(f"w: {model.weight.item()}, b: {model.bias.item()}")
\end{verbatim}

{\color{blue}
运行结果:
w: 1.9752709865570068, b: 0.0727066770195961
}

代码解释:简单的线性回归模型。

(4)
\begin{verbatim}
import torch
import torch.nn as nn

# 定义一个简单的卷积神经网络
class SimpleCNN(nn.Module):
    def __init__(self):
        super(SimpleCNN, self).__init__()
        self.conv1 = nn.Conv2d(1, 16, 3, 1)
        self.fc1 = nn.Linear(16 * 26 * 26, 10)

    def forward(self, x):
        x = torch.relu(self.conv1(x))
        x = x.view(-1, 16 * 26 * 26)
        x = self.fc1(x)
        return x

# 创建模型
model = SimpleCNN()

# 打印模型架构
print(model)
\end{verbatim}

{\color{blue}
运行结果:
SimpleCNN(
  (conv1): Conv2d(1, 16, kernel\_size=(3, 3), stride=(1, 1))
  (fc1): Linear(in\_features=10816, out\_features=10, bias=True)
)
}

代码解释:使用卷积神经网络 (CNN)。

(5)
\begin{verbatim}
import torch
import torch.nn as nn

# 定义一个简单的LSTM模型
class SimpleLSTM(nn.Module):
    def __init__(self, input_size, hidden_size, output_size, num_layers):
        super(SimpleLSTM, self).__init__()
        self.lstm = nn.LSTM(input_size, hidden_size, num_layers, batch_first=True)
        self.fc = nn.Linear(hidden_size, output_size)

    def forward(self, x):
        h0 = torch.zeros(num_layers, x.size(0), hidden_size)  # 初始化隐藏状态
        c0 = torch.zeros(num_layers, x.size(0), hidden_size)  # 初始化记忆状态
        out, _ = self.lstm(x, (h0, c0))
        out = self.fc(out[:, -1, :])  # 取最后一个时间步的输出
        return out

# 设置参数
input_size = 10
hidden_size = 20
output_size = 1
num_layers = 2

# 创建LSTM模型
model = SimpleLSTM(input_size, hidden_size, output_size, num_layers)

# 随机生成输入数据
input_data = torch.randn(5, 3, input_size)  # (batch_size, sequence_length, input_size)
output = model(input_data)

print(output)
\end{verbatim}

{\color{blue}
运行结果:
tensor([[ 0.0036],
        [ 0.0132],
        [-0.0052],
        [ 0.0087],
        [-0.0096]], grad\_fn=<AddmmBackward0>)
}

代码解释:LSTM(长短期记忆网络)的实现。

(6)
\begin{verbatim}
import torch
import torch.nn as nn

# 自定义损失函数
class MyLoss(nn.Module):
    def __init__(self):
        super(MyLoss, self).__init__()

    def forward(self, input, target):
        loss = torch.mean((input - target) ** 2)
        return loss

# 使用自定义损失函数
loss_fn = MyLoss()
input = torch.tensor([1.0, 2.0], requires_grad=True)
target = torch.tensor([1.5, 2.5])

loss = loss_fn(input, target)
print(loss.item())
\end{verbatim}

{\color{blue}
运行结果:
0.25
}

代码解释: 自定义损失函数。

(7)
\begin{verbatim}
import torch

# 检查是否有GPU
device = torch.device("cuda" if torch.cuda.is_available() else "cpu")

# 创建张量并移动到GPU
x = torch.randn(3, 3).to(device)
y = torch.randn(3, 3).to(device)

# 进行加法运算
z = x + y

# 将结果移动回CPU并打印
print(z.cpu())
\end{verbatim}

{\color{blue}
运行结果:
tensor([[-0.9987, -0.6456,  1.2761],
        [-1.7489,  1.5891, -1.6832],
        [ 0.6497, -1.3672,  0.8532]])
}

代码解释:GPU加速计算。

(8)
\begin{verbatim}
import torch
import torch.nn as nn

# 定义一个简单的RNN模型
class SimpleRNN(nn.Module):
    def __init__(self, input_size, hidden_size, output_size):
        super(SimpleRNN, self).__init__()
        self.rnn = nn.RNN(input_size, hidden_size, batch_first=True)
        self.fc = nn.Linear(hidden_size, output_size)

    def forward(self, x):
        h0 = torch.zeros(1, x.size(0), 20)  # 初始化隐藏状态
        out, _ = self.rnn(x, h0)
        out = self.fc(out[:, -1, :])  # 取最后一个时间步的输出
        return out

# 创建RNN模型
model = SimpleRNN(input_size=10, hidden_size=20, output_size=1)

# 随机生成输入数据
input_data = torch.randn(5, 3, 10)  # (batch_size, sequence_length, input_size)
output = model(input_data)

print(output)
\end{verbatim}

{\color{blue}
运行结果:
tensor([[ 0.7989],
        [-0.0113],
        [-0.0014],
        [ 0.3521],
        [ 0.8133]], grad\_fn=<AddmmBackward0>)
}

代码解释:RNN(循环神经网络)的实现。


  \section{实验总结}
性能与调试分析让我意识到了代码效率的重要性。通过使用cProfile和timeit,我能够识别出代码中的性能瓶颈,并通过优化循环和算法来显著提升执行速度。同时,pdb调试工具的使用让我能够更直观地理解代码的执行流程,这对于发现和修复逻辑错误非常有帮助。元编程的学习则让我对Python的灵活性和强大功能有了新的认识。我学会了如何使用装饰器来增强函数的功能,而无需修改其内部实现。这种元编程技术不仅提高了代码的可维护性,也让我对如何构建更智能和灵活的代码有了更深的理解。通过构建和训练神经网络模型,我不仅学到了深度学习的基础知识,还体会到了机器学习的强大潜力。PyTorch的动态计算图让我能够灵活地定义模型结构,并且易于调试和修改。在数据加载和预处理方面,我学会了如何使用PyTorch的DataLoader和transforms来高效地处理大规模数据集。
总的来说,今天的学习经历不仅让我掌握了新的技术知识,更重要的是,我学会了如何将这些知识应用到实际问题中,解决实际问题。我期待在未来的学习中,能够将这些技能应用到更复杂的项目中,解决更多实际问题。

\end{document}